\section{Compression schemes}

Various combinations of the previously detailed techniques are implemented in the codec. The encoder will try for all possible techniques and write to file the most advantageous method. They are listed here:

\begin{itemize}
	\item \textbf{RLE-Gb}: the image is RLE-encoded then compressed with Exp-Golomb. A LUT also needs to be transmitted to allow the mapping of shortest codewords to most probable symbols.
	
	\item \textbf{RLE-Ath}: follows the same process as the RLE-Gb but the Golomb code is replaced by the arithmetic encoder. The symbols and their probabilities also need to be transmitted in a LUT.
	
	\item \textbf{MTF-Ath}: the image is transformed using the MTF and then is compressed by the arithmetic encoder. The dictionnary of the MTF and the probabilities for the arithmetic coder need to be added to the file.
	
	\item \textbf{Benzid}: the method is inspired from the one proposed in \cite{benzid}. The image is transformed using the MTF then the TRE and compressed using the arithmetic coder.
\end{itemize}

Moreover, all the techniques are tested on the image in two scanning orders: horizontal and vertical.

In the compressed file, a flag identifies the chosen method. For some of them, additional size information is also needed to correctly parse the file while decoding. The LUTs are always transmitted with fixed size numbers (8 or 32 bits numbers depending on the case). The typical structure of the compressed file is shown below:

\begin{table}[h]
	\centering
	\begin{tabular}{|c|c|c|c|}
		\hline
		\multicolumn{2}{|c|}{Header}                                                                                     & Data                                                      & LUT                                                                        \\ \hline
		\begin{tabular}[c]{@{}c@{}}Method \\ flag\end{tabular} & \begin{tabular}[c]{@{}c@{}}Size \\ of data\end{tabular} & \begin{tabular}[c]{@{}c@{}}Compressed\\ image\end{tabular} & \begin{tabular}[c]{@{}c@{}}Dictionnaries\\ for reconstruction\end{tabular} \\ \hline
	\end{tabular}
\end{table}